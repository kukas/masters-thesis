
\chapter{Discussion}



\section{On the baseline selection}

All three papers introduced in the previous section utilize two types of baselines to show the improvements of their proposed methods. The first type of baseline is a direct comparison to the existing multilingual language model from which their experiments originate. For all papers it is the original XLM-R model from \citet{conneau_unsupervised_2020}, \citet{chung_improving_2020} additionaly compares their model to mBERT \cite{devlin_bert_2019}. The second type of baseline is obtained by retraining the original XLM-R model with the same training procedures as the proposed methods to control for variables that may influence the results, other than the tokenizer training method. In particular, \citet{chung_improving_2020} specifies that they control for the hyperparameters, number of pretraining languages and number of parameters of the model which as they point out is strongly correlated with the performance of the multilingual model. \cite{conneau_unsupervised_2020}

One difference that the three papers have in common is that they use a higher $\alpha$ scaling factor for the language balance than the original XLM-R paper for the baseline recreation. \citet{zheng_allocating_2021} uses $\alpha=0.7$ while \citet{chung_improving_2020,liang_xlm-v_2023} use $\alpha=0.5$ for their reproduction of XLM-R. This makes the data used for training the model and crucially the tokenizer more biased towards the high-resource languages compared to the original paper, where the factor $\alpha=0.3$ is selected. This choice may affect the balance of the languages in the tokenizer and in turn affect the multilingual performance of the model.

We propose to compare the reproduced methods with the original method from \citet{conneau_unsupervised_2020} by setting the scaling factor $\alpha=0.3$. Moreover, we also propose a second baseline, where we simply use the balancing factor $\alpha=0.0$, which leads to the tokenizer that should be even less biased towards any language. 


- we have shown the importance of the data balance for the language representation quality
    - Rust et al uses alpha=0.7 from mbert
    - Chung uses alpha=0.7 from xlm-r etc

- limitation - we didnt investigate BPE more in depth


- our comparison is limited to 20 langs and small models
    - it is not feasible for us to train Sentencepiece on more than 40M lines of text which limits the number of languages we can compare
    - this is a clear advantage of the Zheng method where they can train a separate tokenizer for each language - and thus overcome this liimtation.
- we also focus heavily on Sentencepiece Unigram as it is the most popular method. From our experiments it seems that Huggingface BPE is a good alternative to Sentencepiece Unigram and it would be interesting to compare it to the other methods

- alpha0 assumes that we have enough data for each language
    - this might not be the case for some extremely-low-resource languages
    - it seems that we need atleast 100k lines to get good tokenizer for that language which is available for most languages
    - but this problem cannot be tackled by any method

- data balance seems to be an extremely large factor for the language representation quality

- Chung has the problem that if low-resource language is assigned to a cluster with high resource, it gets squished
- Zheng seems nice but the downside is training a large amount of tokenizers to be able to measure ALP - this could be overcome with editing the Sentencepiece implementation

- Chung runs the preliminary expeirment over different cluster numbers and finds the 8 to be best. We see that the best tokenizer is yielded with higher number of clusters ? The reason might be that their method doesnt fix the vocabluary size and higher number of clusters means smaller vocabulary size

- strangely, we see that alpha=0.7/0.5 which are the baselines for chung and liang are quite close to our replications with low number of clusters. Low number of clusters is selected in CHung/linag so whats going on there?