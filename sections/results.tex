\chapter{Results}

% What I want to say
%     - we see that there are differences between tokenizers on the metrics
%         - tokenizer method influences vocabulary allocation
%              - algorithm, and even implementation
%              - definitely compare my unigram / huggingface unigram to show the difference
%         - alpha factor influences vocabulary allocation
%         - data size influences vocabulary allocation but converges
%         - tokenizer influences vocabulary overlap but 
%         - data size influences vocabulary overlap but also converges
%     - 
% now with the knowledge of differences, lets compare the Limi unigram, Limi bpe and tokmix
% we see that the metrics correlate with the downstream tasks. Although there are task-specific differences
% - therefore cpt, ar, jsd are good metrics for comparing tokenizers

% - we can see that all balancing methods do influence the quality of the tokenizers on languages
% - we observe that all balancing methods produce similar improvements on lowresource
% - we check this by comparing the balancing methods on downstream and we do not see any significant differences between them. We do see improvements over the strong baseline in some cases.
% - therefore we can conclude that balancing does influence NER and POS tagging although with the comparison to alpha0.3, the improvements are very small

In this chapter, we present our theoretical findings and the results of our experiments. First we validate the usefulness of our proposed vocabulary allocation and vocabulary overlap metrics. We do this on a theoretical level by analytically contrasting our metrics to existing metrics used in the literature. Then, we train three tokenizers using the Huggingface Tokenizers library and assess whether they differ in the metrics we propose. We see that there are significant differences between the three tokenizers. Then we look at how these differences manifest, when we use these tokenizers to train otherwise identical multilingual models. We observe that the differences in tokenizers are reflected in the learned representations of the models. Especially when considering word-level downstream tasks, the learned contextualized representations are better when the tokenizer segments the text into longer tokens (high characters per token), when the tokenizer allocates tokens more uniformly (high average rank) and when there is less overlap between the languages (high Jensen-Shannon divergence).

Then we switch to the original Sentencepiece library, which is used in the related work we replicate \cite{conneau_unsupervised_2020,chung_improving_2020,zheng_allocating_2021,liang_xlm-v_2023}. We see that there are differences in the implementation of the Unigram and BPE algorithms between the two libraries and that the Huggingface library yields subpar Unigram tokenizers. We further assess other factors that influence the tokenizers quality, such as the total training data size, the size of the alphabet and most importantly the language imbalance in the training dataset. We observe that we need around 100k-1M lines per language to train a good, multilingual tokenizer. The alphabet size does affect the number of UNK tokens but does not have a significant influence on the rest of the metrics if we stay in the range of 1000-5000 alphabet size. As for the language imbalance, we observe that it does influence the per-language metrics heavily and that it lowers the tokenization quality for the low-resource languages more than it improves it for the high-resource. The last batch of experiment becomes the main baseline we use for comparing the next experiments.

Finally, we replicate the works of \citet{chung_improving_2020,zheng_allocating_2021,liang_xlm-v_2023} and create several variations of their tokenizers that aim to improve the text segmentations especially for the low-resource languages. We carefully compare these replicated tokenizers with the traditional method of training Sentencepiece Unigram tokenizers on a joint, multilingual corpus. 

While the beforementioned authors do compare their methods to the standard tokenizer training, the crucial difference in our methodology is that we compare their replicated language-balancing methods with the vanilla Sentencepiece tokenizer trained on a \textbf{balanced} multilingual dataset. We find that when compared to this stronger, yet simple baseline, the replicated approaches do not yield significant improvements on our proposed tokenizer metrics per-language nor in overall.

\xxx{maybe rewrite the previous two paragraphs, it is a bit confusing? Now on the second read it doesnt seem so bad}

We then validate our findings by training a second batch of masked language models that differ only in the tokenizer used. We choose the clustering method of \citet{chung_improving_2020} and allocation method of \citet{zheng_allocating_2021} and compare them with Sentencepiece Unigram tokenizers trained on balanced and unbalanced data. We observe that the replicated methods of Chung, Zheng and balanced Unigram do improve the language modeling capabilities over the unbalanced baseline on the word-level tasks while having no impact in the sentence-level task. On the other hand, we do not observe significant differences between the replicated methods and the simple baseline of training the tokenizer on a balanced set which is in line with our previous assessment where we used our proposed tokenizer metrics. Our findings suggest that in our scaled down setting of 20 languages and 120k vocabulary size, the simpler method of training a tokenizer on a joint, balanced corpus is sufficient to create a good multilingual vocabulary that represents all languages well.

\section{Analytical comparison of the metrics}

In this section we compare the metrics we propose to the metrics used in the literature. We focus on each of our proposed metrics and draw connections to the metrics used in the literature.
% compare the average tokenized length \cite{chung_improving_2020,liang_xlm-v_2023}, average log probability \cite{zheng_allocating_2021} and we also draw comparison between our average rank and standard information entropy.

\subsection{Characters per token}

The CPT metric is connected to the average tokenized length (description length) metric used in \citet{chung_improving_2020,liang_xlm-v_2023}. The authors suggest using the metric to compare whether one tokenizer splits a selected low-resource language into more tokens compared to another tokenizer. The average tokenized length is defined as the average number of tokens per sentence: 
\begin{equation}
    TL(\tau, C_l) = \frac{\sum_{s \in C_l}|\tau(s)|}{|C_l|}
\end{equation}

The tokenized length can be expressed as the product of the reciprocal of CPT metric and the average sentence length, which is language-specific and not dependent on the tokenizer:
\begin{equation}
\label{eq:tl}
    CPT(\tau, C_l)^{-1} \cdot \frac{\sum_{s \in C_l}|s|}{|C_l|} = \frac{\sum_{s \in C_l}|\tau(s)|}{\sum_{s \in C_l}|s|} \cdot \frac{\sum_{s \in C_l}|s|}{|C_l|} = \frac{\sum_{s \in C_l}|s|}{|C_l|} = TL(\tau, C_l)
\end{equation}

Even though the metrics are equivalent \xxx{is this the word?}, we use the CPT metric instead of the average tokenized length because we believe it is more intuitive (higher CPT is better) and it is easier to interpret thanks to the lower bound of 1 character per token.

CPT is also similar to another metric used in the literature --- the word fertility metric used in \citet{rust_how_2021}. The word fertility is defined as the average number of tokens per word. We can see that the same argument as in \autoref{eq:tl} can be made about fertility and CPT. If we consider a language-specific constant "average number of characters per word", we see that fertility and CPT are proportional. The fertility metric has been shown to correlate with downstream performance and therefore seems to be a good metric. The downside of this metric is that it is not language-agnostic because it is not defined for languages without word delimiters such as Chinese or Thai. 

\subsection{Average rank}

Now, we would like to address how does our \textit{average rank} compare to different metrics used in the literature. 

First we examine the \textit{average log probability} (ALP) defined in \citet{zheng_allocating_2021}. This metric is proposed for the same purpose as our AR metric. It is also said to measure language-specific vocabulary capacity and the authors claim that it is "penalized by the subword units with low-frequency". Surprisingly, we can show that the ALP metric is equivalent to the product of negative entropy and average tokenized length.

\xxx{todo fix the notation}
The ALP is defined as follows. Given a monolingual corpus composed of sentences $C_l = {s_1, ..., s_{|C_l|}}$ from the language $l$ and tokenized with vocabulary $V_\tau$, the average log probability is defined as follows:

\begin{equation}
    ALP(\tau, C_l) = \frac{1}{|C_l|} \sum_{j=1}^{|C_l|} \sum_{k=1}^{|s_j|} \log p_{uni}(s_{kj})
\end{equation}

We can simplify the formula by observation that the sum over the log token probabilities over the tokens of the corpus can be expressed as the sum over the unique tokens in the tokenizer vocabulary, provided we have the counts of the tokens in the corpus. We can then express the ALP metric as follows: \xxx{the equation is ugly and i guess the count definition is not needed}

\begin{align}
    ALP(\tau, C_l) &= \frac{1}{|C_l|} \sum_{t \in V_\tau} count(t, \tau(C_l)) \log \hat{p}_{\tau(C_l)}(t) \\
    \text{where}~count(t, \tau(C_l)) &= \sum_{j=1}^{|C_l} \sum_{k=1}^{|s_j|} \mathbbm{1}[s_{kj} == t]
\end{align}

From here we introduce the total number of tokens in the corpus $|C_l|$ and we can rewrite the ALP metric as follows:

\begin{align}
ALP(\tau, C_l) &= \frac{1}{|C_l|} \sum_{t \in V_\tau} count(t, \tau(C_l)) \log \hat{p}_{\tau(C_l)}(t) \\
&= \frac{1}{|C_l|} \sum_{t \in V_\tau} \frac{\sum_{t' \in \tau(C_l)} count(t', \tau(C_l))}{\sum_{t' \in \tau(C_l)} count(t', \tau(C_l))} count(t, \tau(C_l)) \log \hat{p}_{\tau(C_l)}(t) \\
&= \frac{\sum_{t' \in \tau(C_l)} count(t', \tau(C_l))}{|C_l|} \sum_{t \in V_\tau} \frac{count(t, \tau(C_l))}{\sum_{t' \in \tau(C_l)} count(t', \tau(C_l))} \log \hat{p}_{\tau(C_l)}(t) \\
&= \frac{\sum_{t' \in \tau(C_l)} count(t', \tau(C_l))}{|C_l|} \sum_{t \in V_\tau} \hat{p}_{\tau(C_l)}(t) \log \hat{p}_{\tau(C_l)}(t) \\
&= - TL(\tau, C_l) \cdot H(\hat{p}_{\tau(C_l)})
\end{align}

\xxx{the last derivation needs "lemma" that $\sum_{t' \in \tau(C_l)} count(t', \tau(C_l)) == \sum_{s \in C_l}|\tau(s)|$}

% By analysis of ALP we see that the metric is composed of two metrics that are already well defined. Strangely, the metrics are multiplied together even though the effects of the metrics go against each other. On one hand the lower the token length the better (lower tokenized sentence length means longer and more meaningful tokens), on the other hand arguably, the higher entropy the better (more uniform distribution is better than very unbalanced distribution). It is not clear why would we want to multiply these two metrics together. The authors do not provide this analysis and discussion. \xxx{now interpretaion occured to me: the best ALP is achieved when we have short tokens that are balanced. So actually it could be interesting. But still the authors do not provide this analysis.}

By examination of ALP, it becomes evident that the metric is a composition of two already well-established metrics. Interestingly, these metrics are multiplied together, even though they seem to be inversely related. On the one hand, shorter token lengths are generally considered to be better (as a shorter tokenized sentence length means longer and more meaningful tokens), while on the other, a higher entropy is often deemed more desirable (a more uniform distribution is preferrable to a more skewed one). One interpretation could be that high ALP is achieved when the vocabulary consists of a large number of short tokens that are similarly useful (have a uniform distribution). The authors, unfortunately, do not provide an analysis or discussion to shed light on this aspect.

\subsection{Average rank and entropy}

Next we will compare the Average Rank and information entropy. The entropy of the tokenized corpus is defined as follows:

\begin{equation}
    H(\tau(C_l)) = - \sum_{t \in V_\tau} \hat{p}_{\tau(C_l)}(t) \log \hat{p}_{\tau(C_l)}(t)
\end{equation}

We see that entropy provides similar characteristics as AR in the sense that more uniform distributions result in higher entropy and more tokens dedicated to a language also result in higher entropy. As we can see, the formula for entropy and AR differ only in the sign and one of the multiplied terms. The sign is not important as we are only interested in the relative values of the metrics. The difference in the multiplied terms is that AR uses the rank of the token, while entropy uses the log-probability of the token. 

To proceed with the analysis, we will assume that the tokens follow the Zipf's distribution. This is a reasonable assumption for natural language data and empirically it holds \xxx{add image}. The Zipf's distribution is defined as follows:

\begin{equation}
    p_{zipf}(t) = \frac{1}{rank(t) \cdot H_{|V_\tau|}}
\end{equation}

Where $H_{|V_\tau|}$ is the $|V_\tau|$-th harmonic number used as a normalization constant.

Taking the logarithm of $p_{zipf}(t)$, we get:

\begin{equation}
    \log p_{zipf}(t) = - \log rank(t) - \log H_{|V_\tau|}
\end{equation}

Now if we plug in the log-probability of the token into the entropy formula and leave out the constant as we are interested only in the relative values with a fixed vocabulary size, we get:

\begin{equation}
    H(\tau(C_l)) = \sum_{t \in V_\tau} \hat{p}_{\tau(C_l)}(t) \log rank(t)
\end{equation}

We see that under the Zipfian assumption, the entropy may be viewed as a "Average Log Rank". This may be empirically observed in the Figure \ref{fig:ar_alr_entropy}.

\begin{figure}
    \centering
    \includegraphics[width=0.9\textwidth]{img/temp/ar_alr_entropy.jpg}
    \caption{Average Rank vs Average Log Rank vs Entropy}
    \label{fig:ar_alr_entropy}
\end{figure}

This provides us with an intuitive understanding of the difference between the two metrics. AR and entropy can be viewed as being related, with the difference being in their sensitivity to the rank of the tokens. AR, being directly related to rank, is more sensitive to changes in probability in lower-frequency tokens. This is because the weighted mean used in AR is more affected by linear rank values than the logarithmic rank values used in entropy calculation.

\subsection{Jensen-Shannon divergence}

Overlap in tokenization has been studied in \citet{wu_beto_2019}, although the metric used there was the absolute number of overlapping tokens. The benefit of using Jensen-Shannon divergence for measuring the vocabulary overlap is that the metric takes into account the occurence of the shared tokens. This is important because some of the overlapping tokens may be for example infrequent emojis or other special tokens that do not carry much information about the language nor the actual overlap in more meaningful tokens.

\citet{chung_improving_2020} uses the Wasserstein distance (or the "earth mover's distance") to measure the overlap between two languages. We believe this is a mistake as the Wasserstein distance is defined for probability measures (probability distributions on a given metric space). The tokenizer vocabulary has no metric structure and the authors of the article do not specify how they define the metric on the vocabulary. It is therefore not suitable to use the Wasserstein distance for measuring the overlap between two tokenizers.


\section{The influence of tokenizer metrics on the language representation quality}
\label{sec:influence_of_metrics}

In this section, we present some of the results from \citet{limisiewicz_tokenization_2023}. We empirically establish that the metrics we propose are useful for assessing the differences between tokenizers. We also show that the differences in tokenizers are reflected in the learned representations of the models. 

For this experiment, we train three distinct tokenizers --- Huggingface Unigram, Huggingface BPE and TokMix using the Huggingface Tokenizers library. The experimental setup is explained in detail in \xxx{ref}.
 % The training data is the CC100 corpus sampled with the smoothing factor $\alpha=0.25$ which corresponds to the XLM-R tokenizer (there $\alpha=0.3$). For all tokenizers we choose the default  In the \ref{tab:20l_metrics} we see the overall macro average metrics for the three tokenizers. 

\begin{table}
\caption{In the first batch of experiments, we compare the Huggingface Unigram, Huggingface BPE, and our TokMix method based on merging Huggingface Unigram tokenizers. Huggingface Unigram has significantly lower vocabulary allocation scores ($-0.4$ CPT and $-153$ AR) than BPE and TokMix. This means that Unigram uses shorter tokens and the capacity of the vocabulary is used less uniformly. Moreover, the vocabulary has more overlap (JSD decrease by $-0.03$) between the languages for Unigram.  The scores are macro averages over all languages, computed over a holdout portion of the CC100 corpus. We sample 10k lines from each language which we have empirically found to be enough to get representative results.}
\label{tab:20l_metrics}
\begin{tabular}{lrrrrr}
\toprule
Tokenizer & Alphabet & \# UNKs & CPT & AR & JSD \\
\midrule
huggingface\_bpe\_alpha0.25 & 1000 & 14040.1 & 3.713 & 1253.7 & 0.783 \\
TokMix\_alpha0.25 & 2497 & 1203.2 & 3.691 & 1163.4 & 0.773 \\
huggingface\_unigram\_alpha0.25 & 12616 & 4.5 & 3.204 & 1010.5 & 0.745 \\
\bottomrule
\end{tabular}
\end{table}


In the \autoref{tab:20l_metrics} we see that the choice of the tokenization method largely influences the vocabulary allocation and overlap metrics. The Huggingface BPE tokenizer produces the longest tokens, the most uniform allocation of tokens and the least overlap between languages. On the other hand, the Huggingface Unigram segments the text into shorter tokens and the average vocabulary overlap between all languages is much higher. The high overlap might be related to the low allocation as it is more likely that shorter tokens are shared between languages. Because we use the default Huggingface settings for the tokenizer training, the resulting tokenizers differ also in the vocabulary size. As we will discuss in \autoref{sec:implementation} and \autoref{sec:character_coverage}, this difference is not sufficient to explain the gap between the tokenizers. % Nevertheless the difference in the alphabet size has an influence on the Chinese language. As we discuss in \citet{limisiewicz_tokenization_2023}, on the language level, the Unigram Tokenizer allocates more tokens to Chinese (as measured by the Average Rank metric) compared to the other methods. This is probably related to the alphabet size, where the majority of the low-occurence tokens might come from the less frequent logograms of the Chinese dictionary.

Next, we use the tokenizers from \autoref{tab:20l_metrics} and pretrain three masked language models with otherwise identical configuration and training data. We then evaluate the quality of the output representations of these models by probing the contextualized vectors against several tasks (XNLI, UD, POS, NER). We test the in-language and cross-lingual settings. The in-language setting are shown in \autoref{fig:pair_analysis_20L}. We show the relationship between our metrics and the downstream tasks. The probe performance is largely dependent on the tested language, to assess only the differences yielded by the use of different tokenizers, we center the per-language results on the mean across the three models. This way we see only the improvements and deteriorations of the models compared to the mean performance. The metrics are centered in the same way. 

We see that the differences between tokenizers are reflected in the representation quality. High CPT and AR metrics are correlated with better probe performance especially on the word-level tasks such as part of speech tagging, named entity recognition and dependency labeling. Moreover the cross-lingual performance is also correlated with the JSD metric as shown in \autoref{fig:X_pair_analysis_20L}.

\xxx{maybe I can actually get rid of the scatterplots and just show the correlation tables from the paper. The problem is the mixing of 6 langs and 20 langs which I do not explain here.}

All correlations are summarized in \autoref{tab:corr_in_lang_20l} and \autoref{fig:corr_x_lang_20l}. The length of the tokens has a strong positive influence on POS, dependency labeling, and NER results ($r > 0.65$), while it does not significantly affect NLI results. The correlation between the average rank and NER scores is weaker but still significant. Moreover, it is significantly correlated with XNLI accuracy with a medium coefficient $r = 0.56$. 

\xxx{what to do with JSD, in paper we present the 6 lang experiments. Here I wanted to keep things simple and focus on the 20 lang experiments. But it seems that for 20 langs the influence of JSD is actually positive for NER and there is $\approx 0$ influence on XNLI. }

Note that in this results section we have included only a part of the experiments from our paper \citet{limisiewicz_tokenization_2023}. Namely we focus only on the experiments that are based on the same 20 languages as the rest of the thesis. For our purposes we focus only on the more general conclusions. We refer the reader to the original paper for the full set of experiments, where we additionaly discuss in more detail which tasks are more affected by which properties of the tokenizers.

% Firstly, we see that the default settings of the Huggingface tokenizers produce very different alphabet sizes and in turn affect the UNK rate. As we will see later, this does not affect the metrics significantly.


\begin{figure}[H]
    \centering
    \includegraphics[width=\textwidth]{paper/figures/pair_analysis_20L.pdf}
    \caption{We compare the tokenizer metrics against the contextualized representation quality. For each tokenizer we pretrain a masked language model, freeze it and train a linear probe for each task and each of the available languages. We observe high spearman correlation between CPT and the word-level tasks (NER, POS, UD) and high correlation between AR and the sentence-level task XNLI. This suggests that our vocabulary allocation metrics are good indicators of the tokenizers quality and higher vocabulary allocation leads to better downstream performance. Each data point corresponds to an average result over three seeds of probe training and evaluating on one of the languages. The results for each language are centered around the mean to account for the differences between languages.}
    \label{fig:pair_analysis_20L}
\end{figure}

\begin{figure}[H]
    \centering
    \includegraphics[width=\textwidth]{img/temp/X_pair_analysis_20L.png}
    \caption{We compare the tokenizer metrics against the cross-lingual performance of the models. For each tokenizer we pretrain a masked language model, freeze it and train a linear probe on each of the available languages. Then we evaluate the models on all languages the probe has \textbf{not} been trained on, assessing the cross-lingual properties of the model. Here we observe high correlation between JSD and the word-level tasks, especially the POS and UD. This suggests that less overlap (higher divergence) between the vocabularies of the languages leads to better cross-lingual performance for the word-level tasks.
    \xxx{remove the ar and cpt metrics?}} \xxx{For XNLI, we see low correlation with a marginal p-value.}
    \label{fig:X_pair_analysis_20L}
\end{figure}

\begin{table}
\centering

\begin{tabular}{lccc}
\toprule
 & \multicolumn{2}{c}{\bf{V. Allocation}} & \bf{MLM} \\
 & (AR) &  (CPT)  &  (MRR) \\
\midrule
CPT    &    \bf{0.790} &     - &  - \\
MRR    &  \bf{-0.723} &  \bf{-0.913} &  - \\
NER    &   \bf{0.394} &   \bf{0.657} &  \bf{-0.745} \\
POS    &     0.320 &   \bf{0.724} &  \bf{-0.754} \\
Dep l. &     0.266 &   \bf{0.675} &  \bf{-0.695} \\
NLI   &    \bf{0.56} &    0.388 &  \bf{-0.437} \\ 
\bottomrule
\end{tabular}
\caption{Spearman correlations between centered in-language task results and tokenizer measures. Statistically significant correlations ($p<0.01$) are bolded. Computed for 20 languages.}
\label{tab:corr_in_lang_20l}
\end{table}

\begin{figure}[H]
    \centering
    \includegraphics[width=\textwidth]{img/temp/corr_x_lang_20l.png}
    \caption{Correlations between task cross-lingual transfer results and tokenization measures."Stars denote statistical significance: (* coresponeds to $p<0.05$ and ** to $p<0.01$).\xxx{remove the ar and cpt metrics? Merge with the previous correlation table?}}
    \label{fig:corr_x_lang_20l}
\end{figure}

% ---------

\section{The choices that influence the tokenizers quality}

\subsection{Implementation}
\label{sec:implementation}

\begin{table}
\caption{In this table, we compare the Huggingface and Sentencepiece implementations of the Unigram and BPE algorithms. The Huggingface Unigram tokenizer is a clear outlier in terms of all metrics. We can see that this is a problem in the implementation as the corresponding Sentencepiece tokenizer \textit{unigram\_alpha0.25\_alphabet} scores much higher. Interestingly, we found that the BPE implementation (\textit{huggingface\_bpe\_alpha0.25}) is better in Huggingface than in Sentencepiece (\textit{bpe\_alpha0.25}).}
\label{tab:hugg_vs_sentpiece}
\begin{tabular}{lrrrrr}
\toprule
Tokenizer & Alphabet & \# UNKs & CPT & AR & JSD \\
\midrule
huggingface\_bpe\_alpha0.25 & 1000 & 14040.1 & 3.713 & 1253.7 & 0.783 \\
unigram\_alpha0.3 & 2666 & 923.5 & 3.702 & 1190.7 & 0.768 \\
bpe\_alpha0.25 & 1215 & 7235.6 & 3.666 & 1212.9 & 0.774 \\
unigram\_alpha0.25\_alphabet & 12577 & 4.5 & 3.629 & 1125.5 & 0.767 \\
huggingface\_unigram\_alpha0.25 & 12616 & 4.5 & 3.204 & 1010.5 & 0.745 \\
\bottomrule
\end{tabular}
\end{table}


As shown in the previous \autoref{sec:influence_of_metrics}, we observe that the Huggingface Unigram tokenizer leads to significantly worse metrics than the other tokenizers. We investigate this difference by turning to the original Sentencepiece implementation of the algorithm and running a comparable experiment with the different library. For comparison, we also train a comparable BPE tokenizer using the Sentencepiece library. All tokenizers are trained with vocabulary size 120\,000.

The results are presented in \autoref{tab:hugg_vs_sentpiece}. We see that the implementation has an effect on the tokenization. We compare the Sentencepiece Unigram tokenizer (\textit{unigram\_alpha0.25\_alphabet}) trained on the same data and with similar parameters as the Huggingface Unigram (\textit{huggingface\_unigram\_alpha0.25}). We see that there is a large difference between the two. Huggingface Unigram underperforms all of our tokenizers. On the other hand the Sentencepiece Unigram approaches the BPE tokenizers. We further see that if we restrict the vocabulary size for the Sentencepiece Unigram (\textit{unigram\_alpha0.3}), we close the gap between Unigram and BPE. 

Interestingly, the Huggingface implementation of BPE (\textit{huggingface bpe alpha0.25}) seems to be better than the Sentencepiece implementation of BPE (\textit{bpe alpha0.25}) in our experimental setup, yielding higher vocabulary allocation metrics (CPT and AR).

Because of the subpar implementation of the Unigram algorithm in the Huggingface library, we use the Sentencepiece implementation for the rest of the experiments.

\subsection{Data size}
\label{sec:data_size}

\begin{table}
\centering
\caption{We measure how much data is generally needed for the tokenizer training. We train handful of Sentencepiece Unigram tokenizers on different amounts of balanced multilingual data. We observe that after 100k-1M lines per language, the tokenizers converge to similar vocabulary allocation and overlap scores. The significance of this experiment is that we find out experimentally how much data is needed for the tokenizer training and we can use this information to make sure that we provide enough data for each language for the further experiments.}
\label{tab:data_size_influence}
\begin{tabular}{rrrrrr}
\toprule
 Lines per language &  Alphabet size &  Number of UNKs &      CPT &          AR &      JSD \\
\midrule
               1000 &           3598 &          520.35 & 3.301636 &  958.414048 & 0.765687 \\
              10000 &           4725 &          117.75 & 3.597563 & 1089.112498 & 0.765236 \\
             100000 &           5041 &           65.55 & 3.695797 & 1192.201089 & 0.767133 \\
            1000000 &           5079 &           62.60 & 3.702038 & 1204.659073 & 0.767357 \\
            1500000 &           5176 &           55.90 & 3.705119 & 1210.664835 & 0.767348 \\
            2000000 &           5180 &           56.35 & 3.705109 & 1212.489940 & 0.767327 \\
\bottomrule
\end{tabular}
\end{table}


We are interested in how much data is needed for the tokenizer training. For this experiment, we sample the same number of lines ($1\times10^{3}, 1\times10^{4}, 1\times10^{5}, 1\times10^{6}, 1.5\times10^{6} \text{ and } 2\times10^{6} \text{ per language}$) for each language from the CC100 corpus. Then we train Sentencepiece Unigram tokenizers on different amounts of data and compare the results in \autoref{tab:data_size_influence}. We see that the metrics improve with the amount of data, but the improvement is not substantial after 100k-1M lines per language. We use these results as a rule of thumb for the rest of the experiments and where possible, we use at least 100k but preferrably 1M lines per language for the rest of the experiments.

\subsection{Character coverage}
\label{sec:character_coverage}

\begin{table}
\caption{We check the tradeoff of including a large alphabet size. We train Sentencepiece Unigram tokenizers with a different target character coverage and observe the resulting alphabet size, number of UNKs and tokenizer metrics. We observe that the alphabet size grows with the coverage and the number of UNKs decreases, as expected. We observe that at both extremes of the character coverage parameter, the vocabulary allocation decreases. The results indicate that the alphabet size between 1000 and 5000 provides a good tradeoff between the number of UNKs and the allocation metrics, while including all characters in the alphabet does not come with a significant decrease in the allocation metrics (-0.05 CPT).}
\label{tab:coverage_influence}
\begin{tabular}{lrrrrr}
\toprule
Coverage & Alphabet & \# UNKs & CPT & AR & JSD \\
\midrule
98.0\% & 539 & 17386.5 & 3.631 & 1115.3 & 0.749 \\
99.5\% & 1136 & 7786.9 & 3.702 & 1173.1 & 0.765 \\
99.95\% & 2678 & 910.6 & 3.705 & 1196.7 & 0.768 \\
99.995\% & 4813 & 83.0 & 3.695 & 1188.7 & 0.769 \\
99.9995\% & 8226 & 10.2 & 3.678 & 1164.2 & 0.769 \\
100.0\% & 13658 & 1.9 & 3.650 & 1124.1 & 0.768 \\
\bottomrule
\end{tabular}
\end{table}


Because we observe the large difference in the alphabet size between Huggingface tokenizers (implied by the use of the default trainer settings of the library), we investigate the influence of this parameter on our tokenizer metrics. We train the Sentencepiece Unigram tokenizers with different character coverage (98\%, 99.5\%, 99.95\%, 99.995\%, 99.9995\%, and 100.0\%) on data sampled with $\alpha=0.3$ from CC100. We show the results in \autoref{tab:coverage_influence}. The character coverage parameter determines, how many distinct Unicode characters are included in the vocabulary of the tokenizer. As expected, we see that there is a direct relationship between the character coverage, alphabet size and the number of unknown tokens in the validation set. We also see that our metrics are not largely affected by the alphabet size, as the alphabet accounts for at most 10\% of the whole vocabulary size 120\,000. The lowest vocabulary allocation metrics are on the extremes of the character coverage, where the alphabet size is either very small or very large. We assume that the small alphabet size leads to a large number of unknown tokens and the tokenizer is forced to segment words containing characters outside of the alphabet, as these unknown tokens might even be characters with diacritics. In the range of 1000-5000 alphabet size, we see that the metrics are not largely affected by the alphabet size. On the other extreme, we suspect that the large alphabet size starts to take up a larger portion of the vocabulary and the tokenizer has less capacity for longer tokens which we observe as a lower overall CPT and AR. We note that including all characters in the training set does not come with a large decrease in our tokenizer metrics (-0.05 CPT and -70 AR compared to 99.95\% coverage). For later experiments, we use the Sentencepiece default character coverage of 99.95\%. When comparing tokenizers with different alphabet sizes, we are aware of the fact that the metrics might be affected by the alphabet size and we take this into account when interpreting the results.

\section{Tokenizer training with data imbalance}
\label{sec:tokenizer_training_with_data_imbalance}

\begin{table}
\caption{Data balance experiments.}
\label{tab:data_balance_metrics}
\begin{tabular}{lrrrrr}
\toprule
Tokenizer & Alphabet & \# UNKs & CPT & AR & JSD \\
\midrule
unigram\_alpha0.0 & 2975 & 617.1 & 3.712 & 1212.9 & 0.767 \\
unigram\_alpha0.3 & 2666 & 923.5 & 3.702 & 1190.7 & 0.768 \\
unigram\_alpha0.5 & 2859 & 729.0 & 3.618 & 1143.8 & 0.769 \\
unigram\_alpha0.7 & 2733 & 883.2 & 3.556 & 1107.1 & 0.770 \\
unigram\_alpha1.0 & 2476 & 1286.3 & 3.442 & 1041.8 & 0.772 \\
\bottomrule
\end{tabular}
\end{table}


In this section, we present the tokenizers that become the baselines for comparing the related works we replicate. Our experiments with the data imbalance follow the original tokenizer training recipe from XLM-R and mBERT. The method is training the tokenizers on a corpus created by combining monolingual data in different proportions. On one extreme we have the $\alpha=1.0$, where all data available for each language is combined. On the other we have $\alpha=0.0$, where the data is sampled per-line from each language with the same probability. 

The contribution of our method is investigating how the language imbalance affects each language individually using the metrics we have proposed. Moreover, we investigate how the other proposed methods relate to all settings of $\alpha$, not only to the unbalanced settings of $\alpha=0.5 \textrm{ or } 0.7$ as is done in \citet{chung_improving_2020,zheng_allocating_2021,liang_xlm-v_2023}.

For the data balancing experiment, we train 5 tokenizers with vocabulary size of 120\,000 on an increasingly imbalanced corpus with $\alpha = 0.0, 0.3, 0.5, 0.7, 1.0$ using the CC100 with 20 languages. For $\alpha = 0.0, 0.3, 0.5, 0.7$ we make sure to sample at least 100k lines per language as we have found this to be important in \autoref{sec:data_size}. We note that the data imbalance for $\alpha=1.0$ is so large, we needed to settle for 30k-70k training lines for the five least resourceful languages (ka, ur, te, mr, sw) because of memory constraints. We use the Sentencepiece Unigram tokenizer with the default settings. Specifically, we use the default character coverage 99.95\%. We evaluate the tokenizers on a balanced validation set sampled from holdout portion of the CC100 corpus. 

The overall results with metrics macro averaged over the 20 languages are presented in \autoref{tab:data_balance_metrics}. The results demonstrate a clear disparity in the quality of tokenization depending on the data balance. Training on balanced data leads to higher overall metrics than on unbalanced data. The imbalance also affects the alphabet size as it is possible to cover 99.95\% of the characters in the training data with smaller alphabet because of overrepresentation of few high-resource languages.

We explore the reason for the decreasing performance of the tokenizers trained on imbalanced data in \autoref{fig:data_balance_vs_allocation_per_lang}. We plot the differences in vocabulary allocation metrics between the most unbalanced tokenizer $\alpha=1.0$ and the rest of the tokenizers. We sort the languages by the data size available starting with the highest-resource languages and ending with the low-resource. We see that the vocabulary allocation metrics for the high resource languages (en, vi, ru, fr, es, th) are improving with the data imbalance. On the other hand, the low resource languages (ka, ur, te, mr, sw) are disproporionally more affected by the imbalance and their vocabulary allocation metrics are decreasing significantly. This suggests that the marginal benefit of adding more data to the high resource languages is lower than the incurred cost on the quality of tokenization for the low resource languages. We see this tradeoff in the \autoref{tab:data_balance_metrics} as a overall decrease in the average CPT and AR.

\begin{figure}[H]
    \centering
    \includegraphics[width=\textwidth]{figures/ar_cpt_vs_alpha.pdf}
    \caption{We examine the impact of the language imbalance on the Sentencepiece Unigram tokenizer training. We train five tokenizers with an increasing language imbalance controlled by the $\alpha$ parameter. Then we look at the effect on the vocabulary allocation metrics per language. We center the results using the most unbalanced tokenizer with $\alpha=1.0$. As expected, the more balanced tokenizers have higher vocabulary allocation scores for low resource languages and lower scores for high resource languages. Interestingly, the effect varies across languages. For example the vocabulary allocation of high-resource Vietnamese or French is not as affected by the decrease in training data as English or Russian.}
    \label{fig:data_balance_vs_allocation_per_lang}
\end{figure}


\section{Comparison of balancing methods}
\label{sec:comparison_balancing_methods}

\begin{table}
\caption{In this summary table, we present all tokenizers used in this chapter. In the table, we include the tokenizers obtained by replicating the papers \citet{chung_improving_2020,zheng_allocating_2021,liang_xlm-v_2023} in our setting. We also include the Huggingface tokenizers from \autoref{tab:20l_metrics} and Sentencepiece Unigram tokenizers from \autoref{tab:data_balance_metrics}. As we can see, the Huggingface Unigram tokenizer is a clear outlier in terms of all metrics even after taking in account the higher alphabet size as explored in \autoref{tab:coverage_influence}. Further, we can see that the clustering methods with a higher number of clusters are improving over the baselines the authors used (\textit{Unigram, $\alpha$=0.5} and \textit{Unigram, $\alpha$=0.7}). On the other hand, we see that using more balanced data for training the Sentencepiece Unigram (\textit{Unigram, $\alpha$=0.0}) leads to better overall performance compared to the replicated methods. We note that the alphabet sizes for all relevant tokenizers stay in the stable range of 1000-5000 so we do not expect this variable to influence the tokenizer metrics. The rows are sorted by the CPT score.}
\label{tab:all_tokenizers_metrics}
\begin{tabular}{lrrrrr}
\toprule
Tokenizer & Alphabet & \# UNKs & CPT & AR & JSD \\
\midrule
Hugg. BPE, $\alpha$=0.25 & 1000 & 14040.1 & 3.713 & 1253.7 & 0.783 \\
Unigram, $\alpha$=0.0 & 2975 & 617.1 & 3.712 & 1212.9 & 0.767 \\
Chung 20 clusters & 4123 & 270.3 & 3.702 & 1098.7 & 0.766 \\
Unigram, $\alpha$=0.3 & 2666 & 923.5 & 3.702 & 1190.7 & 0.768 \\
TokMix, $\alpha$=0.25 & 2497 & 1203.2 & 3.691 & 1163.4 & 0.773 \\
Chung 16 clusters & 3933 & 387.1 & 3.677 & 1102.2 & 0.767 \\
Liang 20 clusters & 3709 & 341.4 & 3.676 & 1103.2 & 0.765 \\
Zheng 20langs & 4854 & 245.7 & 3.673 & 1094.5 & 0.765 \\
Liang 16 clusters & 3655 & 416.8 & 3.669 & 1106.2 & 0.767 \\
Sentpiece. BPE, $\alpha$=0.25 & 1215 & 7235.6 & 3.666 & 1212.9 & 0.774 \\
Unigram, $\alpha$=0.5 & 2859 & 729.0 & 3.618 & 1143.8 & 0.769 \\
Chung 8 clusters & 4870 & 684.4 & 3.575 & 1061.1 & 0.770 \\
Unigram, $\alpha$=0.7 & 2733 & 883.2 & 3.556 & 1107.1 & 0.770 \\
Chung 4 clusters & 3253 & 648.6 & 3.546 & 1071.9 & 0.768 \\
Liang 8 clusters & 4283 & 568.2 & 3.544 & 1081.6 & 0.767 \\
Liang 4 clusters & 3698 & 419.2 & 3.512 & 1082.5 & 0.769 \\
Unigram, $\alpha$=1.0 & 2476 & 1286.3 & 3.442 & 1041.8 & 0.772 \\
Hugg. unigram, $\alpha$=0.25 & 12616 & 4.5 & 3.204 & 1010.5 & 0.745 \\
\bottomrule
\end{tabular}
\end{table}


In this section, we compare the tokenizers we replicated following the works of \citet{chung_improving_2020,zheng_allocating_2021,liang_xlm-v_2023}. For each of the clustering methods (Chung, Liang), we create 4 variants with different number of clusters (4, 8, 16, 20) as that is the main hyperparameter of the methods. For Zheng, we create one variant by greedily maximizing the ALP metric across the 20 languages. We present the results alongside the rest of the experiments in this chapter to provide context and be better able to interpret the differences between the methods.

In \autoref{tab:all_tokenizers_metrics}, we compare overall metrics for all tokenizer experiments\footnote{Note that for Sentencepiece Unigram with $\alpha=0.0\text{ and }0.3$ we retrain the tokenizers on 20M and 10M lines of data respectively (compared to 2M and 5M from \autoref{tab:data_balance_metrics}) to match the data provided to the replicated methods. This does not increase the metrics by a lot as observed in \autoref{sec:data_size}. Nevertheless we wanted to match the data sizes to fix this variable.}. We sort the table by the CPT metric. We see that the unbalanced tokenizers ($\alpha=1.0, 0.7$) are placed at the bottom of the results along with the underperforming Huggingface Unigram implementation. We see that generally, the standard tokenizers trained on a balanced dataset provide the best results on our tokenizer metrics. Next we see the clustering methods of Chung and Liang with 20 and 16 clusters along with the Zheng method. Only then we see the standard tokenizers trained on less balanced or unbalanced dataset and clustering methods with lower number of clusters. We visualize our vocabulary allocation metrics on a scatterplot in \autoref{fig:all_tokenizers_AR_vs_CPT} to better observe the differences between the tokenizers and explore the relationship between CPT and AR. Each point on the scatterplot is one tokenizer and its position is determined by the CPT and AR metrics. We connect related experiments with a line and color-code them. Here we can see that the tokenizers with high CPT often have high AR. Nevertheless, we also have tokenizers with low AR and high CPT but never the other way around. Our intuition is that it is not possible to construct a tokenizer with high number of useful tokens which are all very short. With the context of different tokenization methods, we can see the degree of Huggingface Unigram underperformance.

\begin{figure}[H]
    \centering
    \includegraphics[width=\textwidth]{figures/all_tokenizers_AR_vs_CPT.pdf}
    \caption{We visualize the overall vocabulary allocation metrics for all tokenizers from Table \ref{tab:all_tokenizers_metrics}. We observe that the vocabulary allocation scores are related --- higher AR usually means higher CPT.  We also observe that Huggingface Unigram is a clear outlier, although combination of separate, monolingual Huggingface Unigrams (TokMix) approaches the performance of the Sentencepiece Unigram with the corresponding data imbalance ($\alpha=0.3$). We see that the balancing methods overperform the unbalanced Unigrams ($\alpha=1.0$, $\alpha=0.7$) in terms of CPT but perform similarly or worse to the simple case of running the Sentencepiece Unigram trainer on a balanced set $\alpha=0.0$.}
    \label{fig:all_tokenizers_AR_vs_CPT}
\end{figure}

Crucially, we see that the reproduced methods of \citet{chung_improving_2020,zheng_allocating_2021,liang_xlm-v_2023} do improve over the unbalanced baselines $\alpha=1.0, 0.7, 0.5$, especially with higher number of clusters but do not outperform the simple case of training the Sentencepiece Unigram on a balanced dataset $\alpha=0.0$. We also observe that the clustering methods with higher number of clusters along with Zheng are close to each other on the CPT-AR plot. We assume this is because with higher $k$, the clustering methods reduce to the Zheng method (training separate tokenizers for each language). Similarly, with lower number of clusters, the methods are closer to the vanilla Sentencepiece Unigram trained on a unbalanced dataset.

\begin{figure}[H]
    \centering
    \includegraphics[width=\textwidth]{figures/all_tokenizers_AR_vs_JSD.pdf}
    \caption{We visualize the tokenizers from Table \ref{tab:all_tokenizers_metrics} in terms of Average Rank and Jensen-Shannon Divergence. Here we can see that all methods based on Sentencepiece result in similar overlap independent of the allocation. This is interesting because the replicated balancing methods (Chung, Zheng, Liang) work by splitting the data and training separate tokenizers. Nevertheless, after merging the separate subtokenizers they all seem to end up with similar vocabulary overlaps. The highest vocabulary isolation is surprisingly achieved by the Huggingface BPE tokenizer, which is contrary to the hypothesis stated by \citet{chung_improving_2020,zheng_allocating_2021} that the tokenizers trained on the concatenation of all data tend to select subwords shared across all languages.}
    \label{fig:all_tokenizers_AR_vs_JSD}
\end{figure}

We also explore the relationship between vocabulary allocation and vocabulary overlap on the \autoref{fig:all_tokenizers_AR_vs_JSD}\footnote{The CPT-JSD plot is similar but is less readable therefore we present the relationshop between AR and JSD}. We see that the differences between all methods based on Sentencepiece are small compared to the differences between Huggingface tokenizers. This is surprising because one of the motivations for the method of Chung and Liang is to promote overlap between similar languages while minimizing overlap between distant languages. We would therefore expect the overall overlap to decrease, as the number of spurious token sharing decreases. Nevertheless we observe that the resulting overlap is lower or similar to our Sentencepiece Unigram tokenizers. Similarly, Zheng method works by training separate tokenizers for each language and then combining them into one. We would expect the overlap to be even lower than the original Sentencepiece Unigram but we observe that the overlap is comparable. 
Surprisingly, the lowest vocabulary overlap (highest JSD) is achieved by the Huggingface BPE. 

\begin{figure}[H]
    \centering
    \includegraphics[width=\textwidth]{figures/zheng_vs_alphas.pdf}
    \caption{We zoom into the results of the Zheng method and compare the vocabulary allocation across the individual languages represented by this tokenizer against the backdrop of the vanilla Unigram tokenizers trained with different data imbalances from \ref{fig:data_balance_vs_allocation_per_lang}. We observe a striking similarity between the vocabulary allocation of the Zheng tokenizer and the Unigram tokenizer with $\alpha=0.0$, especially in terms of characters per token. This comes as a large surprise because the Zheng method works by training a separate tokenizer for each language and then merging them together. Despite the different method of obtaining the vocabulary, the resulting tokenizers are very similar across the languages.}
    \label{fig:zheng_vs_alphas}
\end{figure}

We investigate the differences between the balanced Unigram and the replicated methods in more detail by examining the CPT and AR metrics computed per language. We use the data balance experiments from \autoref{sec:tokenizer_training_with_data_imbalance} for comparison. We start by comparing the Zheng method with the increasingly imbalanced Unigram tokenizers in \autoref{fig:zheng_vs_alphas}. We plot the increase or decrease in vocabulary allocation metrics for each language sorted by the data available. Remarkably, we see that the Zheng method is similar in terms of CPT and AR per language to the Unigram tokenizer trained on the balanced set $\alpha=0.0$. The similarity seems to be higher in the CPT metric although the AR metric is also very similar especially for the highest and lowest resource languages. We find this result quite surprising because of the distinctness of the Zheng method --- it trains a separate tokenizer for each language and then merges them together. Nevertheless, the resulting tokenizers are very similar to the Unigram tokenizer trained on the balanced set. This result seems to validate the choice of the ALP metric for the selection of vocabulary sizes in the Zheng method (see \autoref{fig:zheng_vs_alphas_alp}) but it also calls into question the necessity of the separate training of tokenizers for each language. One advantage of the Zheng method is that by splitting the data into separate languages it lowers the memory requirements for the training of the separate Sentencepiece Unigram although the required compute-time is higher because of the need for training hundreds of tokenizers to be able to select the ones that maximize the overall ALP after merging them.


\begin{figure}[H]
    \centering
    \includegraphics[width=0.7\textwidth]{figures/zheng_vs_alphas_alp.pdf}
    \caption{Intrigued by the similarity between the Zheng tokenizer and the Unigram tokenizer with $\alpha=0.0$ from Figure \ref{fig:zheng_vs_alphas} we also look at the ALP metric which is used for the selection of vocabulary sizes in the Zheng method. Here we see that the greedy optimization of ALP across languages indeed results in a similar vocabulary allocation as the Unigram tokenizer with $\alpha=0.0$.}
    \label{fig:zheng_vs_alphas_alp}
\end{figure}

\begin{figure}[H]
    \centering
    \includegraphics[width=\textwidth]{figures/chung_vs_alphas.pdf}
    \caption{Here we inspect the language-level vocabulary allocation of the Chung method. Similarly to the Zheng method, the Chung method also performs similarly to the Unigram tokenizer with $\alpha=0.0$. Unfortunately, we believe this is an artifact of the choice of our training data for the Chung method. We use a balanced dataset ($\alpha=0.0$) for training the cluster-specific tokenizers. The balance of the data seems to be more important than the clustering step. After merging the cluster-specific tokenizers, the resulting tokenizer is very similar to the Unigram tokenizer with $\alpha=0.0$.}
    \label{fig:chung_vs_alphas}
\end{figure}

Next we inspect the Chung method and compare it in detail to our Unigram tokenizers in \autoref{fig:chung_vs_alphas}. For comparison we select a run with low number of clusters (k=4) and high number of clusters (k=16). We see that the different number of clusters yield different results. In the case of higher number of clusters, we see that the tokenizer exhibits similar trend in CPT and AR across the languages as the balanced Unigram tokenizer with $\alpha=0.0$ albeit with some deviations. In the case of lower number of clusters, the metrics per language seem to be more distinct compared to our Unigram tokenizers. 

\begin{table}
\caption{Cluster assignments for {k} clusters.}
\label{tab:chung_clusters_k16}
\begin{tabular}{lr}
\toprule
Languages & Size \\
\midrule
ar, ur & 14020 \\
tr & 7582 \\
en, fr & 13549 \\
hi, mr & 12232 \\
ru, bg & 13194 \\
vi & 7582 \\
te & 7582 \\
el & 7582 \\
zh & 7582 \\
he & 7582 \\
ta & 7582 \\
sw & 7582 \\
ka & 7582 \\
th & 7582 \\
es & 7582 \\
de & 7582 \\
\bottomrule
\end{tabular}
\end{table}

\begin{table}
\caption{Cluster assignments for {k} clusters.}
\label{tab:chung_clusters_k4}
\begin{tabular}{lr}
\toprule
Languages & Size \\
\midrule
zh, ar, ru, bg & 27425 \\
el, ur, ta, te, th, he, ka & 48841 \\
en, es, tr, sw, vi, fr, de & 43622 \\
hi, mr & 12108 \\
\bottomrule
\end{tabular}
\end{table}


We look at the CPT per language for k=16 more closely and identify the languages where the Chung tokenizer differs significantly from the Unigram tokenizer with $\alpha=0.0$. We see that the CPT drops significantly for Bulgarian (bg), Urdu (ur), Marathi (mr), and to some degree French (fr). On the other hand we see smaller improvements for English (en), Vietnamese (vi), Spanish (es), Thai (th), Hindi (hi), and Tamil (ta). We compare this to the cluster assignments in \autoref{tab:chung_clusters_k16}. Revealingly, we observe that all languages with the large drop in CPT has been assigned to a cluster with another, higher-resource language. Bulgarian is assigned with Russian (8th largest corpus versus 3rd largest corpus), Urdu with Arabic (17th vs. 13th), Marathi with Hindu (18th vs. 14th) and French with English (4th vs. 1st)

We continue with similar analysis for the 4 clusters. We see that the CPT for Bulgarian (bg), Turkish (tr), Arabic (ar), and Swahili (sw) is lower than any of our Unigram tokenizers. On the other hand, we see significant improvements for Thai (th), Hebrew (he), Greek (el), and Hindi (hi) over our Unigram tokenizers. Marathi (mr) achieves highest CPT increase over the unbalanced Unigram baseline, although the increase stays in the range of the $\alpha=0.5$ Unigram tokenizer. We again look at the cluster assignments in \autoref{tab:chung_clusters_k4}. We observe that Bulgarian and Arabic are assigned to a cluster with higher resource Russian (3rd) and Chinese, which could explain the decrease in CPT for the two. Similarly, Swahili and Turkish which use the Latin script are assigned to a cluster with higher resource English (largest corpus) and Vietnamese (2nd largest). On the other hand, we see that that Thai, Hebrew and Greek are assigned to a cluster with lower-resource languages --- Tamil (ta), Georgian (ka), Urdu (ur), and Telugu (te). As we have observed, Thai, Hebrew and Greek benefit from  We observe these lower-resource languages to have a lower CPT than the $\alpha=0.7$ Unigram.

\begin{figure}[H]
    \centering
    \includegraphics[width=\textwidth]{figures/liang_vs_alphas.pdf}
    \caption{We inspect the language-level vocabulary allocation of the Liang method. We see similarities to the Chung method in \autoref{fig:chung_vs_alphas}. The some of the differences seem to be the improved Chinese and Arabic for 4 clusters and worse Hebrew and better Urdu for 16 clusters. In overall the results are similar.}
    \label{fig:liang_vs_alphas}
\end{figure}

We look at the \citet{liang_xlm-v_2023} replication results in \autoref{fig:liang_vs_alphas}. We see that despite a slightly different clustering method and per-cluster vocabulary size selection, the Liang method exhibits similar patterns we observed in the Chung method.

Overall, we infer that the Chung and Liang methods are sensitive to the cluster assignments. Because the training data are merged per-cluster, if low resource languages gets assigned to a cluster with high-resource language, the language imbalance acts in favor of the high-resource language. Bearing this in mind, we know from our experiments in \autoref{sec:tokenizer_training_with_data_imbalance} presented previously, that the benefit of adding more data to the high resource language is lower than the cost that incurs on the low-resource language. We believe this is the cause of the lower overall CPT and AR for the clustering methods.


\section{Comparison of balancing methods on downstream tasks}

To validate our previous assessment from \autoref{sec:comparison_balancing_methods}, we select two replicated methods by \citet{chung_improving_2020,zheng_allocating_2021}. For the clustering method, we select a low- and high- number of clusters $k=4\text{ and }16$. We compare these replicated tokenization methods to the standard Unigram tokenizers trained on differently balanced dataset with $\alpha=1.0, 0.3\text{ and }0.0$

For each of the six tokenizers, we train a masked language model and probe it on three tasks - natural language inference (NLI), part of speech tagging (POS) and named entity recognition (NER). 

\footnote{Note that even though the probe training is done with three different random seeds, the model pretraining was done only once. The variance between pretraining runs is therefore not measured and the error bars should be interpreted as such.}

% We observe the most clear regularity in cross-language performance for word-level tasks (NER and POS), where all balancing methods improve over the unbalanced $\alpha=1.0$ model. Next we see higher POS in-language scores for the Chung methods and higher NER in-language scores for the balanced unigrams ($\alpha=0.0\text{ and }0.3$). For in-language NLI, we do not see any systematic effect. The seemingly random differences between $\alpha=1.0, 0.3\text{ and }0.0$ point to

\begin{figure}
    \centering
    \begin{subfigure}{.5\textwidth}
      \centering
      \includegraphics[width=\linewidth]{figures/probe_overall_inlanguage.pdf}
      \caption{In-language results}
      \label{fig:probe_overall_inlanguage}
    \end{subfigure}%
    \begin{subfigure}{.5\textwidth}
      \centering
      \includegraphics[width=\linewidth]{figures/probe_overall_crosslanguage.pdf}
      \caption{Cross-language results}
      \label{fig:probe_overall_crosslanguage}
    \end{subfigure}
    \caption{We select the replicated methods by \citet{chung_improving_2020,zheng_allocating_2021} and compare them with the vanilla Unigram tokenizers. For comparison, we choose the unbalanced Unigram tokenizer with $\alpha=1.0$ and then two stronger baselines with $\alpha=0.0$ and $\alpha=0.3$ trained on more balanced data. We then pretrain masked language models that differ only in the tokenizer they use and assess the performance of these models on the downstream tasks using probing. We test two settings --- in-language performance, where the model is trained on each of the available languages and then evaluated on the same language, and cross-language performance, where the model is also trained on each language but evaluated on all \textit{but} the training language. The results are a macro average over all the languages (in-language results) or all language pairs (cross-language results). For each model, language and task we do 3 probe training runs with different random seeds. The error bars represent one standard deviation computed with bootstrapping by randomly sampling seeds for each language.}
    \label{fig:probe_overall}
\end{figure}

% \begin{figure}[H]
%     \centering
%     \includegraphics[width=\textwidth]{img/temp/probe_overall_inlanguage.png}
%     \caption{probe overall inlanguage}
%     \label{fig:probe_overall_inlanguage}
% \end{figure}


\begin{figure}[H]
    \centering
    \includegraphics[width=\textwidth]{figures/probe_detailed_inlanguage_over_baseline.pdf}
    \caption{We zoom in on the in-language results from Figure \ref{fig:probe_overall_inlanguage} and compare the performance of the balanced tokenizers against the unbalanced Unigram tokenizer with $\alpha=1.0$ over all tested languages for the tasks. In case of the word-level tasks, especially in the case of named entity recognition, we observe a clear trend in line with our tokenizer investigations in \ref{fig:chung_vs_alphas}. The balancing methods improve the language representations for the word-level tasks. For the sentence-level tasks, we do not observe any systematic effects. This might be in part due to the fact that the NLI task does not include 4 of our low-resource languages. The error bands are one standard deviation computed from the three probe training runs with different random seeds.}
    \label{fig:probe_overall_inlanguage_over_baseline}
\end{figure}


\begin{figure}[H]
    \centering
    \includegraphics[width=\textwidth]{figures/probe_detailed_crosslanguage_over_baseline_lang_tgt.pdf}
    \caption{Here we investigate in detail the cross-lingual results from Figure \ref{fig:probe_overall_crosslanguage} with comparison to the unbalanced Unigram tokenizer with $\alpha=1.0$. We observe that word-level task transfers behave in line with the tokenizer investigations in \ref{fig:chung_vs_alphas}. Moreover it seems that both high-resource and low-resource languages benefit from the balancing methods, although the change is most clear at the low-resource side. For the sentence-level tasks, we do not observe any systematic effects.}
    \label{fig:probe_overall_crosslanguage_over_baseline}
\end{figure}



\begin{figure}[H]
    \centering
    \includegraphics[width=\textwidth]{figures/probe_detailed_inlanguage_scattermatrix.pdf}
    \caption{We visualize the in-language results from Figure \ref{fig:probe_overall_inlanguage} in a scatter matrix. We center the results for each language and then plot the differences from mean performance against the differences in our tokenizer metrics. We see significant spearman correlations for the NER and POS tasks, although for POS the correlation is low. For the NLI task, we do not observe any significant correlations.}
    \label{fig:probe_overall_inlanguage_scattermatrix}
\end{figure}


% \begin{figure}[H]
%     \centering
%     \includegraphics[width=\textwidth]{img/temp/probe_overall_crosslanguage.png}
%     \caption{probe overall crosslanguage}
%     \label{fig:probe_overall_crosslanguage}
% \end{figure}

\begin{figure}[H]
    \centering
    \includegraphics[width=\textwidth]{figures/probe_detailed_crosslanguage_scattermatrix.pdf}
    \caption{We visualize the cross-language results from Figure \ref{fig:probe_overall_crosslanguage} in a scatter matrix. We center the results for each language and then plot the differences from mean performance against the differences in the vocabulary overlap metric (JSD). We see significant, very low negative correlation for the NER and POS tasks. This might suggest that the word-level tasks benefit only very slightly from an increase in overlap (decrease in JSD). For the NLI task, we do not observe any significant correlations.}
    \label{fig:probe_overall_crosslanguage_scattermatrix}
\end{figure}

% visualization idea:

% - visualization of tokenizer balance difference is too noisy to see the differences between methods
%     - smoothing num_lines_per_language vs cpt
%     - or fitting a line
%     - something that highlights if one balancing method is better than another

% \section{Preliminary experiments}
% \subsection{The importance of training data size}
% \subsection{Differences in tokenizer implementations}
% \section{Reproduction of baselines}
% \section{Document-level clustering method}
% \section{Extrinsic evaluation}


% - replication of the previous work
%     - Chung
%         - reproductions in Overlap-based Vocabulary Generation Improves Cross-lingual Transfer Among Related Languages

%     - Liang
%         - they use 900k vocab, they compare their model to XLM-R which is not fair!
%             - they discuss it in section 6.4
%         - reproductions: https://github.com/stefan-it/xlm-v-experiments
%             - For XQuAD they did not reproduce the improvements
%             - For MasakhaNER they reproduced the improvements
%             - TODO: could use bootstrapping to show whether the improvements are significant

% replications of Chung, Liang
% https://github.com/stefan-it/xlm-v-experiments

% - our beta experiments point to the randomness of the output - the smooth sweep across the beta values seems to produce quite noisy outpus
