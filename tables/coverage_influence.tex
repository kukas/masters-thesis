\begin{table}
\centering
\caption{We check the tradeoff for including a large alphabet size. We train Sentencepiece Unigram tokenizers with a different target character coverage and observe the resulting alphabet size, number of UNKs and tokenizer metrics. We observe that the alphabet size grows with the coverage and the number of UNKs decreases, as expected. With growing alphabet size, the vocabulary has less capacity to include longer tokens, which is reflected in the decrease in allocation metrics after including more than 5000 characters in the alphabet. With the smallest alphabet, the number of UNKs is high which negatively impacts CPT as the tokens are artificially split by the unknown characters. The results indicate that the alphabet size between 1000 and 5000 provides a good tradeoff between the number of UNKs and the allocation metrics, while including all characters in the alphabet does not come with a significant decrease in the allocation metrics (-0.05 CPT).}
\label{tab:coverage_influence}
\begin{tabular}{lrrrrrr}
\toprule
 coverage &    vocab &  alphabet &      unk &      cpt &          ar &      jsd \\
    98.0\% & 120000.0 &     539.0 & 17386.50 & 3.631156 & 1115.250107 & 0.748642 \\
\midrule
    99.5\% & 120000.0 &    1136.0 &  7786.95 & 3.701585 & 1173.092070 & 0.764939 \\
   99.95\% & 120000.0 &    2678.0 &   910.65 & 3.705273 & 1196.714334 & 0.768465 \\
  99.995\% & 120000.0 &    4813.0 &    82.95 & 3.695392 & 1188.724073 & 0.768683 \\
 99.9995\% & 120000.0 &    8226.0 &    10.20 & 3.678148 & 1164.241707 & 0.768564 \\
99.99995\% & 120000.0 &   12362.0 &     2.30 & 3.656834 & 1133.239755 & 0.768358 \\
   100.0\% & 120000.0 &   13658.0 &     1.95 & 3.650257 & 1124.126335 & 0.768328 \\
\bottomrule
\end{tabular}
\end{table}
