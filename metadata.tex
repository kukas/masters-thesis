

%%% Choose a language %%%

\newif\ifEN
\ENtrue   % uncomment this for english
%\ENfalse   % uncomment this for czech

%%% Configuration of the title page %%%

\def\ThesisTitleStyle{mff} % MFF style
%\def\ThesisTitleStyle{cuni} % uncomment for old-style with cuni.cz logo
%\def\ThesisTitleStyle{natur} % uncomment for nature faculty logo

\def\UKFaculty{Faculty of Mathematics and Physics}
%\def\UKFaculty{Faculty of Science}

\def\UKName{Charles University in Prague} % this is not used in the "mff" style

% Thesis type names, as used in several places in the title
% \def\ThesisTypeTitle{\ifEN BACHELOR THESIS \else BAKALÁŘSKÁ PRÁCE \fi}
\def\ThesisTypeTitle{\ifEN MASTER THESIS \else DIPLOMOVÁ PRÁCE \fi}
%\def\ThesisTypeTitle{\ifEN RIGOROUS THESIS \else RIGORÓZNÍ PRÁCE \fi}
%\def\ThesisTypeTitle{\ifEN DOCTORAL THESIS \else DISERTAČNÍ PRÁCE \fi}
% \def\ThesisGenitive{\ifEN bachelor \else bakalářské \fi}
\def\ThesisGenitive{\ifEN master \else diplomové \fi}
%\def\ThesisGenitive{\ifEN rigorous \else rigorózní \fi}
%\def\ThesisGenitive{\ifEN doctoral \else disertační \fi}
% \def\ThesisAccusative{\ifEN bachelor \else bakalářskou \fi}
\def\ThesisAccusative{\ifEN master \else diplomovou \fi}
%\def\ThesisAccusative{\ifEN rigorous \else rigorózní \fi}
%\def\ThesisAccusative{\ifEN doctoral \else disertační \fi}



%%% Fill in your details %%%

% (Note: \xxx is a "ToDo label" which makes the unfilled visible. Remove it.)
\def\ThesisTitle{Improving Subword Tokenization Methods for Multilingual Models}
\def\ThesisAuthor{Jiří Balhar}
\def\YearSubmitted{2023}

% department assigned to the thesis
\def\Department{Institute of Formal and Applied Linguistics}
% Is it a department (katedra), or an institute (ústav)?
\def\DeptType{Institute}

\def\Supervisor{Ing. Tomasz Limisiewicz}
\def\SupervisorsDepartment{Institute of Formal and Applied Linguistics}

% Study programme and specialization
\def\StudyProgramme{Computer Science}
\def\StudyBranch{Artificial Intelligence}

\def\Dedication{%
    I would like to thank Tomasz Limisiewicz for the time and effort he put into supervising this thesis. I am grateful for his guidance and insightful discussions, which not only shaped the outcome of this thesis but also, hopefully, refined my research skills.

    I would like to thank Ondřej Dušek for all the friendly and encouraging calls we had before I started working on this thesis. 

    I am grateful to David Mareček for inspiring me to work on this topic and for connecting me with Tomasz.
    
    Finally, I would like to thank my family and friends for their unwavering support and encouragement throughout my long and eventful studies. Special thanks go to my mother and brother for all the "gofs", to my father for sharing with me his endless curiosity about the world and science, and to my fiancée, who has always believed in me and supported me in every way possible.
}

\def\AbstractEN{%
    % \xxx{Abstracts are an abstract form of art. Use the most precise, shortest sentences that state what problem the thesis addresses, how it is approached, pinpoint the exact result achieved, and describe the applications and significance of the results. Highlight anything novel that was discovered or improved by the thesis. Maximum length is 200 words, but try to fit into 120. Abstracts are often used for deciding if a reviewer will be suitable for the thesis; a well-written abstract thus increases the probability of getting a reviewer who will like the thesis.}
    % ABSTRACT IS NOT A COPY OF YOUR THESIS ASSIGNMENT!
    In this thesis, we explore the differences between tokenization methods for multilingual neural language models and investigate their impact on language model representation quality. 
    
    We propose a set of metrics to evaluate the quality of tokenizations. We show that the metrics capture the differences between tokenizers and that they correlate with the downstream performance of multilingual language models.
    
    Then, using our metrics, we assess why is the standard tokenizer training on a multilingual corpus reported to be ineffective for multilingual models. We investigate design choices such as data size, implementation or alphabet size. We identify that the issue might be caused by data imbalance and to solve it we propose to sample tokenizer training data uniformly. 

    We compare the standard tokenizer training with three proposed methods we replicate, that aim to mitigate the same reported issues. We show that the principle behind the improvements of the proposed methods is the same as with the uniform sampling.

    Our findings offer a deeper understanding of tokenization methods for multilingual models. We propose a methodology and guidelines for training multilingual tokenizers. Lastly, we show how to achieve improvements in tokenization without the need for more complex tokenization methods.
}

\def\AbstractCS{%
    V této práci jsou zkoumány rozdíly mezi metodami tokenizace pro vícejazyčné neuronové modely  (multilingual language models) a rovněž jejich vliv na kvalitu jazykového modelu. Je definována sada metrik, které jsou použity pro vyhodnocení kvality tokenizace: pomocí experimentů je demonstrováno, že užité metriky zachycují rozdíly mezi tokenizátory a korelují s výkonem vícejazyčných neuronových modelů. 

    Některé práce věnované vícejazyčné tokenizaci uvádí, že standardní postup trénování tokenizátorů na vícejazyčném korpusu není vhodný pro vícejazyčné modely. Tato práce hledá důvod uvedených problémů. Jako možné příčiny jsou zkoumány velikost dat, implementace nebo velikost abecedy. V práci docházíme k závěru, že problém je pravděpodobně způsoben nevyvážeností dat mezi jazyky a navrhujeme řešení v podobě rovnoměrného vzorkování trénovacích dat tokenizátoru.

    V diplomové práci jsou replikovány tři studie, které se zabývají vylepšením metod vícejazyčné tokenizace a jsou porovnány se standardním trénováním na rovnoměrných datech. Díky porovnání je zjištěno, že princip, který stojí za zlepšením u replikovaných metod, je stejný jako u rovnoměrného vzorkování.

    Výsledky diplomové práce poskytují hlubší vhled do problematiky tokenizace pro vícejazyčné modely. Je navržena metodika a doporučení pro trénování vícejazyčných tokenizérů. Nakonec je ukázáno, jak dosáhnout zlepšení tokenizace bez nutnosti použití složitějších tokenizačních metod.
}

% 3 to 5 keywords (recommended), each enclosed in curly braces.
% Keywords are useful for indexing and searching for the theses by topic.
\def\Keywords{%
    {natural language processing}|{multilingual language models}|{subword tokenization}|{NLP}
}


% If your abstracts are long and do not fit in the infopage, you can make the
% fonts a bit smaller by this setting. (Also, you should try to compress your abstract more.)
% Alternatively, consider increasing the size of the page by uncommenting the
% geometry modification in thesis.tex.
\def\InfoPageFont{}
%\def\InfoPageFont{\small}  %uncomment to decrease font size

\ifEN\relax\else
    % If you are writing a czech thesis, you additionally need to fill in the
    % english translation of the metadata here!
    \def\ThesisTitleEN{\xxx{Thesis title in English}}
    \def\DepartmentEN{\xxx{Name of the department in English}}
    \def\DeptTypeEN{\xxx{Department}}
    \def\SupervisorsDepartmentEN{\xxx{Superdepartment}}
    \def\StudyProgrammeEN{\xxx{study programme}}
    \def\StudyBranchEN{\xxx{study branch}}
    \def\KeywordsEN{%
        \xxx{{key} {words}}
    }
\fi
